%======== BEGIN inserted by add_gemeinsam.rb ======================
\documentclass{beamer}
\usepackage[german]{babel}
\usepackage[T1]{fontenc}
%\usepackage[latin1]{inputenc} 
\usepackage[utf8x]{inputenc}

\usepackage{verbatim} 
\usepackage{listings} 

% \lstset{numbers=left, numberstyle=\tiny, numbersep=5pt} 

%\usetheme{Warsaw}
\usetheme{Hannover}
%\usetheme{PaloAlto}
%\usetheme{JuanLesPins}
%\usetheme{Antibes}
%\usetheme{Shingara}
%\usetheme{Berlin}
%\usetheme{Oxygen}

\usecolortheme{beaver}
% albatross | beaver | beetle |
% crane | default | dolphin |
% dove | fly | lily | orchid |
% rose |seagull | seahorse |
% sidebartab | structure |
% whale | wolverine


\title[Ruby]{Was kann Ruby?}
\author{Sven Suska, Thomas Preymesser}
%\institute{}
\date{2009-Juli-2}


\begin{document}
\lstset{language=Ruby}
\lstset{basicstyle=\small,numbers=none, numberstyle=\tiny, numbersep=5pt}
%======== END inserted by add_gemeinsam.rb ======================
\begin{frame}
  \frametitle{Automatisierte Tests mit Ruby / TestUnit}
  Warum automatisierte Tests?
  \begin{itemize}
    \item Leicht auf Knopfdruck aufzurufen
    \item Solide Basis an Tests bei gr"o"seren Anwendungen
    \item Vertrauen bei Modifikation der Anwendung
    \item Test ''dokumentieren'' Verwendung von Code
    \item Test-Driven-Development (TDD)
  \end{itemize}
\end{frame}

\begin{frame}
  \frametitle{Automatisierte Tests mit Ruby}
  Automatisierte Tests mit Ruby
  \begin{itemize}
    \item<1-> TestUnit (Standard--Bibliothek in Ruby)
    \item<2-> RSpec (separat zu installieren)
  \end{itemize}
\end{frame}

\begin{frame}[containsverbatim]
  \frametitle{Automatisierte Tests mit Ruby / TestUnit}
\lstinputlisting[frame=single,caption={konto.rb}]{code/konto.rb}
\end{frame}

\begin{frame}[containsverbatim]
  \frametitle{Automatisierte Tests mit Ruby / TestUnit}
\lstinputlisting[frame=single,caption={test\_konto.rb}]{code/test_konto.rb}
\end{frame}

\begin{frame}[containsverbatim]
  \frametitle{Automatisierte Tests mit Ruby / TestUnit}
  \lstinputlisting[frame=single,caption={Resultat}]{code/test_result}
\end{frame}

\begin{frame}
  \frametitle{Automatisierte Tests mit Ruby / TestUnit}
  \begin{itemize}
    \item Namen der Testmethoden beginnen mit '\texttt{test\_}'
    \item Reihenfolge der Tests ist {\bf nicht} die Reihenfolge im Source
    \item Reihenfolge der Tests ist {\bf nicht} definiert
    \item Aufruf einer einzelnen Testmethode:\\ '\texttt{ruby test\_konto.rb --name test\_anfangssaldo}'
    \item Mehrere Test--Files k"onnen zu Test--Suites zusammengefa"st werden
    \item GUI-Ausgabe statt nur Text m"oglich
    \item Gem ZenTest (autotest)
  \end{itemize}
\end{frame}

\begin{frame}
  \frametitle{Automatisierte Tests mit Ruby / TestUnit}
  Test::Unit assertions
  \begin{itemize}
    \item assert
    \item assert\_equal
    \item assert\_not\_equal
    \item assert\_nil
    \item assert\_not\_nil
    \item assert\_kind\_of
    \item assert\_raise(Exception) \{ ... \}
    \item assert\_nothing\_raised(Exception) \{ ... \}
    \item assert\_in\_delta
    \item ...
  \end{itemize}
\end{frame}

\begin{frame}[containsverbatim]
  \frametitle{Automatisierte Tests mit Ruby / TestUnit}
  \lstinputlisting[frame=single,caption={setup / teardown}]{code/setupteardown.rb}
\end{frame}

\begin{frame}
  \frametitle{Automatisierte Tests mit Ruby / TestUnit}
Optionale Methoden 
\begin{itemize}
  \item setup
  \item teardown
\end{itemize}
werden vor bzw. nach jedem Test aufgerufen
\end{frame}

\begin{frame}
  \frametitle{Automatisierte Tests mit Ruby / TestUnit}
  Meta-Programming Aspekte
  \begin{itemize}
    \item at\_exit
    \item methods ('test\_')
  \end{itemize}
\end{frame}

%======== BEGIN inserted by add_gemeinsam.rb ======================
\end{document}
%======== END inserted by add_gemeinsam.rb ======================
