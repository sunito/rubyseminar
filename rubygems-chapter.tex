\begin{frame}
  \frametitle{RubyGems}
  Gems sind optional installierbare Bibliotheken f"ur
  Ruby.
  
  Vorbild: CPAN f"ur Perl
 
  Beispiele:
  \begin{itemize}
    \item rmagick f"ur Bildmanipulation
    \item roo zum Lesen von Spreadsheets
    \item rails, Ruby on Rails
    \item ... und mehr als ca. 4000 andere gems
  \end{itemize}
\end{frame}

\begin{frame}
  \frametitle{RubyGems}
  \begin{itemize}
    \item Gems werden standardm"a"sig bei http://www.rubyforge.org gehostet
    \pause
    \item es kann aber auch einfach die <gem>.gem Datei heruntergeladen und anschlie"send installiert werden
    \pause
    \item Gems k"onnen weitere abh"angige Gems automatisch installieren
  \end{itemize} 
\end{frame}

\begin{frame}
  \frametitle{RubyGems}
  Installation:
  \lstinputlisting[language={},frame=single,caption={Gem Installation}]{code/rubygems/installation_result}
\end{frame}

\begin{frame}
  \frametitle{RubyGems}
  Verwendung in Programmen
  \lstinputlisting[frame=single,caption={Anwendung}]{code/rubygems/verwendung.rb}
\end{frame}

\begin{frame}
  \frametitle{RubyGems}
  M"oglichkeiten
  \lstinputlisting[frame=single,caption={Versionen}]{code/rubygems/verwendung2.rb}
\end{frame}

\begin{frame}
  \frametitle{RubyGems}
  \begin{itemize}
  
    \item 'gem query --remote' Liste aller Gems bei Rubyforge
    \item 'gem query --remote --details -n ^roo\$' Listet Details zu Gem 'roo' 
    \item 'gem query --remote --details -n xml' Listet alle Gem Beschreibungen die 'xml' im Namen haben
    \item 'gem install foo'
    \item 'gem install foo --version 1.0.0'
    \item 'gem update'
    \item 'gem uninstall roo'
    \item 'gem cleanup'
    \item 'gem help commands'
    \item 'gem server' (\path{http://localhost:8808/})
  \end{itemize}
\end{frame}

\begin{frame}
  \frametitle{RubyGems}
  Eigene Gems erstellen
  \begin{itemize}
    \item .gemspec Dateien erstellen
    \item Gem 'newgem'
  \end{itemize}
\end{frame}
