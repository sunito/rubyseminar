\begin{frame}{Einführung}
\frametitle{Was ist Ruby?}

\begin{center}
''Actually, I'm trying to make Ruby natural, not simple.'' 
\end{center}
\begin{flushright}
\emph{Yukihiro ''matz'' Matsumoto}
\end{flushright}



\begin{itemize}
\pause \item Sprache ausdrucksstark\pause, ''natürlich''\pause, flexibel\pause, ''schön''
\pause \item Konsequent objektorientiert (''everthing is an object'')
\pause \item Dynamische Typisierung (''duck typing'')
\pause \item Closures, Metaprogrammierung, offene Klassen
\pause \item Nachteile: (noch) relativ geringe Verbreitung\pause,  langsam\pause, 
              schlechte Thread-Unterstutzung
\end{itemize}
\end{frame}

%\begin{frame}[containsverbatim]
\begin{frame}[fragile]
  \frametitle{Hallo Welt}
  \begin{itemize}[<+->]
  \item nil
  \item 17
  \item i = 123
  \item i + 1000
  \item puts i
  \end{itemize}
  \pause
  Syntaktischer Hauptbestandteil in Ruby sind Ausdrücke ($Expression$s).\\
  Jeder Ausdruck hat beim Ausführen einen Wert.\\
  \pause
  %\item[none]<2->
  \begin{lstlisting}
  text = "welt"
  puts text
  \end{lstlisting}
  \pause
  \lstinline|   puts "hallo " + text + "!"|
  

\begin{comment}
  \pause
  \lstinline|   puts "hallo #{text}!"| \hspace{5em}  (``String-Interpolation'')
  %\end{itemize}
\end{comment}

\end{frame}

\begin{frame}[fragile]
  \frametitle{irb}
  \begin{center}
   {\LARGE irb}\\
   
   \bigskip
   "interactive ruby"
   \bigskip
   \bigskip
   \bigskip
   \bigskip
   
  \end{center}
   shellprompt\$ irb

\end{frame}

\begin{frame}[fragile]
  \frametitle{Ausdrücke und Werte}
  \begin{tabular}[t]{l@{\hspace{5em}$\longrightarrow$}l}
    %{\em Ausdruck}      &   {\em Wert }  \\
    \lstinline|1|       &   \lstinline|1|  \\
    \lstinline|1 + 2|   &   \lstinline|3|  \\
    \lstinline|a = 2|   &   \lstinline|2|  \\
    \lstinline|1+2; 10+20; 17+4|  &   \lstinline|21|  \\
    \lstinline|puts 1+2|   &   \lstinline|nil|  \\
  \end{tabular}
  
  \pause
  Es gibt zwei Arten von Werten.
  
  Unmittelbare Werte (sind immer dasselbe Objekt):
  \begin{lstlisting}
  17
  true
  nil
  :symbol
  \end{lstlisting}
  \pause
  Referenzierte Werte:
  \begin{lstlisting}
  17.0
  "hallo"
  []
  Symbol
  \end{lstlisting}
\end{frame}

\begin{frame}[fragile]
  \frametitle{Methoden}
  Methoden-Definition
  \begin{lstlisting}
  def dreifach(wert)
    wert * 3
  end
  \end{lstlisting}
  \pause
  Methoden-Aufruf
  \begin{lstlisting}
  puts dreifach(2) 
  \end{lstlisting}
  $\longrightarrow$ 6
\end{frame}


\begin{frame}[fragile]
  \frametitle{Methoden-Syntax}
  Klammersetzung bei Methoden-Parametern ist optional:
  \begin{lstlisting}
  def dreifach wert 
    wert * 3
  end
  \end{lstlisting}
  \pause
  \begin{lstlisting}
  puts dreifach(2) 
  puts dreifach 2 
  \end{lstlisting}
  \pause
  \begin{lstlisting}
  puts(dreifach(2)) 
  \end{lstlisting}
  
  \pause
  \begin{lstlisting}
  3 + 4 * 5 
  \end{lstlisting}
  \pause
  \begin{lstlisting}
  3 + (4 * 5)
  \end{lstlisting}
\end{frame}


