%======== BEGIN inserted by add_gemeinsam.rb ======================
\documentclass{beamer}
\usepackage[german]{babel}
\usepackage[T1]{fontenc}
%\usepackage[latin1]{inputenc} 
\usepackage[utf8x]{inputenc}

\usepackage{verbatim} 
\usepackage{listings} 

% \lstset{numbers=left, numberstyle=\tiny, numbersep=5pt} 

%\usetheme{Warsaw}
\usetheme{Hannover}
%\usetheme{PaloAlto}
%\usetheme{JuanLesPins}
%\usetheme{Antibes}
%\usetheme{Shingara}
%\usetheme{Berlin}
%\usetheme{Oxygen}

\usecolortheme{beaver}
% albatross | beaver | beetle |
% crane | default | dolphin |
% dove | fly | lily | orchid |
% rose |seagull | seahorse |
% sidebartab | structure |
% whale | wolverine


\title[Ruby]{Was kann Ruby?}
\author{Sven Suska, Thomas Preymesser}
%\institute{}
\date{2009-Juli-2}


\begin{document}
\lstset{language=Ruby}
\lstset{basicstyle=\small,numbers=none, numberstyle=\tiny, numbersep=5pt}
%======== END inserted by add_gemeinsam.rb ======================
\begin{frame}
  \frametitle{Fiber}
  \lstinputlisting[frame=single,caption={aufgaben.rb}]{code/aufgaben.rb}
\end{frame}

\begin{frame}
  \frametitle{Fiber}
  \lstinputlisting[frame=single,caption={Ergebnis}]{code/aufgaben_ergebnis}
\end{frame}

\begin{frame}
  \frametitle{Fiber}
  Fiber
  \begin{itemize}
    \item "ahnlich wie Threads
    \item aber Programmautor legt fest, wann und wo R"uckkehr ins aufrufende Programm erfolgt
    \item auch Semi-Coroutinen genannt
    \item mittels \texttt{require 'fiber'} auch Transfer zu anderen Fiber m"oglich
  \end{itemize}
\end{frame}


\begin{frame}
  \frametitle{Aufruf externer Programme}
  Aufruf externer Programme
  \begin{itemize}
    \item \texttt{puts `tar archiv.tar /home/tp/sourcecode`}
    \item oder \texttt{system('tar archiv.tar /home/tp/sourcecode')}
    \item Fehlercode des Programms in globaler Variable \texttt{\$?}
  \end{itemize}
\end{frame}

\begin{frame}
  \frametitle{IO.popen}
  IO.popen
  \lstinputlisting[frame=single,caption={popen-beispiel.rb}]{code/popen-beispiel.rb}
\end{frame} 

%======== BEGIN inserted by add_gemeinsam.rb ======================
\end{document}
%======== END inserted by add_gemeinsam.rb ======================
