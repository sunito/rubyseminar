\documentclass{beamer}


\usepackage[german]{babel}
\usepackage[T1]{fontenc}
%\usepackage[latin1]{inputenc} 
\usepackage[utf8x]{inputenc}

\usepackage{verbatim} 
\usepackage{listings} 

% \lstset{numbers=left, numberstyle=\tiny, numbersep=5pt} 

%\usetheme{Warsaw}
\usetheme{Hannover}
%\usetheme{PaloAlto}
%\usetheme{JuanLesPins}
%\usetheme{Antibes}
%\usetheme{Shingara}
%\usetheme{Berlin}
%\usetheme{Oxygen}

\usecolortheme{beaver}
% albatross | beaver | beetle |
% crane | default | dolphin |
% dove | fly | lily | orchid |
% rose |seagull | seahorse |
% sidebartab | structure |
% whale | wolverine


\title[Ruby]{Was kann Ruby?}
\author{Sven Suska, Thomas Preymesser}
%\institute{}
\date{2009-Juli-2}



\begin{document}
\lstset{language=Ruby}
\lstset{basicstyle=\small,numbers=none, numberstyle=\tiny, numbersep=5pt}
  
  
  
\begin{frame}[fragile]
  \frametitle{Werte: unmittelbar/referenziert}
  
  \pause
  Es gibt zwei Arten von Werten.
  
  Unmittelbare Werte (sind immer dasselbe Objekt):
  \begin{lstlisting}
  17
  true
  nil
  :symbol
  \end{lstlisting}
  \pause
  Referenzierte Werte:
  \begin{lstlisting}
  17.0
  "hallo"
  []
  Symbol
  \end{lstlisting}
\end{frame} 

\begin{frame}
  \frametitle{Ruby-''Compiler''}
  Rubyscript2exe
  ocra
\end{frame} 

\begin{frame}
  \frametitle{Web-Links}
  RubyKids -- Ruby für Kinder: \path{http://www.rubykids.de/articles/content_index}
\end{frame} 





\end{document}
