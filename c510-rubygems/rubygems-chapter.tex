%======== BEGIN inserted by add_gemeinsam.rb ======================
\documentclass{beamer}
\usepackage[german]{babel}
\usepackage[T1]{fontenc}
%\usepackage[latin1]{inputenc} 
\usepackage[utf8x]{inputenc}

\usepackage{verbatim} 
\usepackage{listings} 

% \lstset{numbers=left, numberstyle=\tiny, numbersep=5pt} 

%\usetheme{Warsaw}
\usetheme{Hannover}
%\usetheme{PaloAlto}
%\usetheme{JuanLesPins}
%\usetheme{Antibes}
%\usetheme{Shingara}
%\usetheme{Berlin}
%\usetheme{Oxygen}

\usecolortheme{beaver}
% albatross | beaver | beetle |
% crane | default | dolphin |
% dove | fly | lily | orchid |
% rose |seagull | seahorse |
% sidebartab | structure |
% whale | wolverine


\title[Ruby]{Was kann Ruby?}
\author{Sven Suska, Thomas Preymesser}
%\institute{}
\date{2009-Juli-2}


\begin{document}
\lstset{language=Ruby}
\lstset{basicstyle=\small,numbers=none, numberstyle=\tiny, numbersep=5pt}
%======== END inserted by add_gemeinsam.rb ======================
\begin{frame}
  \frametitle{RubyGems}
  Gems sind optional installierbare Bibliotheken f"ur
  Ruby.
  
  Vorbild: CPAN f"ur Perl
 
  Beispiele:
  \begin{itemize}
    \item rmagick f"ur Bildmanipulation
    \item roo zum Lesen von Spreadsheets
    \item rails, Ruby on Rails
    \item ... und mehr als ca. 4000 andere gems
  \end{itemize}
\end{frame}

\begin{frame}
  \frametitle{RubyGems}
  \begin{itemize}
    \item Gems werden standardm"a"sig bei http://www.rubyforge.org gehostet
    \pause
    \item es kann aber auch einfach die <gem>.gem Datei heruntergeladen und anschlie"send installiert werden
    \pause
    \item Gems k"onnen weitere abh"angige Gems automatisch installieren
  \end{itemize} 
\end{frame}

\begin{frame}
  \frametitle{RubyGems}
  Installation:
  \lstinputlisting[language={},frame=single,caption={Gem Installation}]{code/installation_result}
\end{frame}

\begin{frame}
  \frametitle{RubyGems}
  Verwendung in Programmen
  \lstinputlisting[frame=single,caption={Anwendung}]{code/verwendung.rb}
\end{frame}

\begin{frame}
  \frametitle{RubyGems}
  M"oglichkeiten
  \lstinputlisting[frame=single,caption={Versionen}]{code/verwendung2.rb}
\end{frame}

\begin{frame}
  \frametitle{RubyGems}
  M"oglichkeiten
  \begin{itemize}
    \item \texttt{gem install roo}
    \item \texttt{gem update}
    \item \texttt{gem uninstall roo}
    \item \texttt{gem cleanup}
    \item \texttt{gem query --remote} Liste aller Gems bei Rubyforge
    \item \texttt{gem query --remote --details -n \^{}roo}\$ Listet Details zu Gem 'roo' 
    \item \texttt{gem query --remote --details -n xml} Listet alle Gem Beschreibungen die 'xml' im Namen haben
    \item \texttt{gem install roo --version 1.0.0}
    \item \texttt{gem help commands}
    \item \texttt{gem server} (\path{http://localhost:8808/})
  \end{itemize}
\end{frame}

\begin{frame}
  \frametitle{RubyGems}
  Eigene Gems erstellen
  \begin{itemize}
    \item .gemspec Dateien erstellen
    \item Gem 'newgem'
  \end{itemize}
\end{frame}

%======== BEGIN inserted by add_gemeinsam.rb ======================
\end{document}
%======== END inserted by add_gemeinsam.rb ======================
