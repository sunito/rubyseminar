\documentclass{beamer}

%\usepackage[german]{babel}
\usepackage[T1]{fontenc}
%\usepackage[latin1]{inputenc} 
\usepackage[utf8x]{inputenc}

\usepackage{verbatim} 
\usepackage{listings} 

% \lstset{numbers=left, numberstyle=\tiny, numbersep=5pt} 

%\usetheme{Warsaw}
\usetheme{Hannover}
%\usetheme{PaloAlto}
%\usetheme{JuanLesPins}
%\usetheme{Antibes}
%\usetheme{Shingara}
%\usetheme{Berlin}
%\usetheme{Oxygen}

\usecolortheme{beaver}
% albatross | beaver | beetle |
% crane | default | dolphin |
% dove | fly | lily | orchid |
% rose |seagull | seahorse |
% sidebartab | structure |
% whale | wolverine


\title[Ruby]{Was kann Ruby?}
\author{Sven Suska, Thomas Preymesser}
%\institute{}
\date{2009-Juli-2}


\newcommand{\ordner}{..}
\usepackage[german]{babel}
\usepackage[T1]{fontenc}
%\usepackage[latin1]{inputenc} 
\usepackage[utf8x]{inputenc}

\usepackage{verbatim} 
\usepackage{listings} 

% \lstset{numbers=left, numberstyle=\tiny, numbersep=5pt} 

%\usetheme{Warsaw}
\usetheme{Hannover}
%\usetheme{PaloAlto}
%\usetheme{JuanLesPins}
%\usetheme{Antibes}
%\usetheme{Shingara}
%\usetheme{Berlin}
%\usetheme{Oxygen}

\usecolortheme{beaver}
% albatross | beaver | beetle |
% crane | default | dolphin |
% dove | fly | lily | orchid |
% rose |seagull | seahorse |
% sidebartab | structure |
% whale | wolverine


\title[Ruby]{Was kann Ruby?}
\author{Sven Suska, Thomas Preymesser}
%\institute{}
\date{2009-Juli-2}



\usepackage{tikz}

\begin{document}
\lstset{language=Ruby}
\lstset{basicstyle=\small,numbers=none, numberstyle=\tiny, numbersep=5pt}

\begin{frame}[fragile]
  \frametitle{Methoden}
  Methoden-Definition
  \begin{lstlisting}
  class Adresse
    attr_accessor :name, strasse_nr, :plz, :ort
  end
  \end{lstlisting}
  \pause
  Methoden-Aufruf
  \begin{lstlisting}
  puts dreifach(2) 
  \end{lstlisting}
  $\longrightarrow$ 6
\end{frame}


\begin{frame}[fragile]
  \frametitle{Methoden-Syntax}
  Klammersetzung bei Methoden-Def ist auch optional:
  \begin{lstlisting}
    def dreifach(wert)
      wert * 3
    end
  def dreifach wert 
    wert * 3
  end
  \end{lstlisting}
  \pause
  \begin{lstlisting}
  puts dreifach(2) 
  puts dreifach 2 
  \end{lstlisting}
  \pause
  \begin{lstlisting}
  puts(dreifach(2)) 
  \end{lstlisting}
  
  \pause
  \begin{lstlisting}
  3 + 4 * 5 
  \end{lstlisting}
  \pause
  \begin{lstlisting}
  3 + (4 * 5)
  \end{lstlisting}
\end{frame}



\begin{frame}[fragile]
  \frametitle{Objekte und Klassen}
  Ruby ist fast vollständig objektorientiert. \\ \pause
  D.h. (fast) alles in Ruby ist ein $Objekt$. \\ \pause
  Jedes $Objekt$ gehört zu einer $Klasse$. \\ 
  $Klasse$n enthalten die $Methode$n ihrer $Objekt$e.\\ \pause
  \vspace{2mm}
  Mit dem Methodenaufruf \path{.class} erfahren wir die Klasse:
  \begin{tabular}[t]{l@{\hspace{5em}$\longrightarrow$}l}
    \lstinline|1.class|       &   \lstinline|Integer|  \\
    \lstinline|"abc".class|   &   \lstinline|String|  \\
    \lstinline|[1,2,3].class| &   \lstinline|Array|  \\
    \lstinline|String.class|  &   \lstinline|Class|  \\
    \lstinline|Class.class|   &   \lstinline|Class|  \\
  \end{tabular}
  Auch $Klasse$n sind $Objekt$e. \\ \pause  
\end{frame}

\begin{frame}[fragile]
  \frametitle{Empfänger von Methoden}
  Methodenaufrufe haben {\bf immer} einen Empfänger ($Receiver$).\\ \pause
  Wenn dieser nicht angegeben ist, wird das aktuelle Objekt (\bls{self)} 
  als Empfänger verwendet.\\
  Ein expliziter Empfänger wird mit \path{.} verknüpft vorangestellt:
  \begin{center}
    \lstinline|receiver.methodenname|
  \end{center}

  \pause \medskip 
  In unseren anfänglichen Beispielen war der Empfänger immer implizit. \\
  Das \path{self}-Objekt war hier \path{main}. \\
  \pause
  Beispiele von Exliziten Empfängern:
  \begin{tabular}[t]{l@{\hspace{5em}$\longrightarrow$\hspace{5pt}}l}
    %{\em Ausdruck}      &   {\em Wert }  \\
    \lstinline|1.class|       &   \lstinline|Integer|  \\
    \lstinline|"abc".upcase|   &   \lstinline|"ABC"|  \\
    \lstinline|[5, 2, 4].max|   &   \lstinline|5|  \\
    \lstinline|10.+ 20|  &   \lstinline|30|  \\
    \lstinline|10 + 20|  &   \lstinline|30|  \\
  \end{tabular}
  
\end{frame}

\begin{frame}[fragile]
  \frametitle{Dynamische Typisierung}
  Der Empfänger bestimmt die tatsächlich aufgerufene $Methode$.\\
  \smallskip
  \begin{tabular}[t]{l@{\hspace{5em}$\longrightarrow$\hspace{5pt}}l}
    %{\em Ausdruck}      &   {\em Wert }  \\
    \lstinline|"abc".reverse|   &   \lstinline|"cba"|  \\
    \lstinline|[1,2,3].reverse|   &   \lstinline|[3,2,1]|  \\
  \end{tabular}
  \pause
  \begin{tabular}[t]{l@{\hspace{5em}$\longrightarrow$\hspace{5pt}}l}
    \lstinline|10.+ 20|  &   \lstinline|30|  \\
    \lstinline|"10".+ "20"|  &   \lstinline|"1020"|  \\
  \end{tabular}
  
  \medskip
  \pause
  Wir können Code schreiben ohne den Typ der Objekte zu kennen ("Duck Typing").\\ 
  \begin{tabular}[t]{l@{\hspace{5em}$\longrightarrow$\hspace{5pt}}l}
    \lstinline|dreifach(10)|       &   \lstinline|30|  \\
    \lstinline|dreifach("hallo")|  &   \lstinline|"hallohallohallo"|  \\
  \end{tabular}
  
\end{frame}


\begin{frame}[fragile]
  \frametitle{Vordefinierte Container-Typen}
  Array
  \begin{itemize}[<+->]
  \item \lstinline|[1,2,3]|
  \item \lstinline|["a","b","c"]|
  \item \lstinline|[1,"hallo",true]|
  \end{itemize}
  
  \bigskip
  Hash
  \begin{itemize}[<+->]
  \item \lstinline|{:a => "wert1", :b => "wert2"}|
  \end{itemize}

  Nur Ruby 1.9:
  \begin{itemize}[<+->]
  \item \lstinline|{a: "wert1", b: "wert2"}|
  \end{itemize}
  
  \pause
  Die geschweiften Klammern können weggelassen werden, wenn keine
  Ambiguitäten auftreten können.

\end{frame}


\begin{frame}[fragile]
  \frametitle{Klassen Vererbung}
  Klassen werden mit \bls{class} definiert:
  \tikz \draw[thick,rounded corners=8pt]
(0,0) -- (0,2) -- (1,3.25) -- (2,2) -- (2,0) -- (0,2) -- (2,2) -- (0,0) -- (2,0);

  \begin{lstlisting}
  
 class MeineEnte 
   def quak
     puts "Quak, quak"
   end
 end
  \end{lstlisting}
  \pause
  Objekt erzeugen mit der Methode \bls{new}:
  \begin{lstlisting}
 ente = MeineEnte.new
 ente.quak
  \end{lstlisting}
  \pause
  \begin{tabular}[t]{l@{\hspace{5em}$\longrightarrow$\hspace{5pt}}l}
    \lstinline|ente.class|              &    \lstinline|MeineEnte|  \\
    \lstinline|ente.is_a?(MeineEnte)|   &    \lstinline|true|  \\
  \end{tabular}
\end{frame}


\begin{frame}[fragile]
  \frametitle{Reflexion}
  Ruby besitzt umfassende Möglichkeiten zur Innenschau.
  
  \smallskip
  \begin{tabular}[t]{l@{\hspace{5pt}$\longrightarrow$\hspace{2pt}}l}
  \lstinline|ente.class|             & \lstinline|MeineEnte|  \\
  \lstinline|ente.respond_to? :quak| & \lstinline|true|  \\
  \end{tabular} \\
  
  \medskip
  \lstinline| ente.methods|     \\ 
  $\longrightarrow$ \lstinline|["inspect", "tap", "clone", ... | \\
  \hspace{6em}       \lstinline| ... "quak", "untaint", "nil?", "is_a?"] |  \\
         
  \medskip    
  \lstinline| MeineEnte.instance_methods| \\ 
  $\longrightarrow$  \lstinline|["inspect", "tap", "clone", ...| \\
  \hspace{6em}       \lstinline| ... "quak", "untaint", "nil?", "is_a?"] |  \\
  
\end{frame}



\begin{frame}[fragile]
  \frametitle{Klassen-Vererbung}
  Eine Superklasse kann hinter \path{\<} angegeben werden:
  \begin{lstlisting}
 class MeinString < String
   def initialize(text)
     super
   end
   def quak
     "#{self} quakt."
   end
 end
  \end{lstlisting}
    
  \pause
  Dann stehen für Objekte der Klasse \path{MeinString} 
  alle Methoden der Klasse \path{String} zur verfügung.\\
  \begin{tabular}[t]{l@{\hspace{5em}$\longrightarrow$\hspace{5pt}}l}
    \lstinline|m = MeinString.new("abc")|  &    \lstinline|"abc"|  \\
    \lstinline|m.quak|  &    \lstinline|"abc quakt."|  \\
    \lstinline|m.upcase|                & \lstinline|"ABC"|  \\
    \lstinline|m.class|                  & \lstinline|MeinString|  \\
    \lstinline|"abc".quak|                  & \lstinline|NoMethodError ...|  \\
  \end{tabular}
\end{frame}

\begin{frame}[fragile]
  \frametitle{Ruby hat offene Klassen}
  Wir können alle bestehenden Klassen verändern.
  \begin{lstlisting}
 class String
   def quak
     "#{self} quakt."
   end
 end
  \end{lstlisting}
  
  
  \pause
  Das wirkt sich auf alle Objekte dieser Klasse aus, \\
  auch dann, wenn sie vorher schon existierten!\\
  \begin{tabular}[t]{l@{\hspace{5em}$\longrightarrow$\hspace{5pt}}l}
    \lstinline|m = "abc"|       &    \lstinline|"abc"|  \\
    \lstinline|m.quak|          &    \lstinline|"abc quakt."|  \\
    \lstinline|m.upcase|                & \lstinline|"ABC"|  \\
    \lstinline|m.class|                  & \lstinline|String|  \\
    \lstinline|"abc".quak|                  & \lstinline|"abc quakt."|  \\
  \end{tabular}
\end{frame}



\end{document}
