 
\documentclass{beamer}

\usepackage[german]{babel}
\usepackage[T1]{fontenc}
\usepackage[utf8x]{inputenc} 

\usepackage{verbatim}
\usepackage{listings} 
% \lstset{numbers=left, numberstyle=\tiny, numbersep=5pt} 
 

\usetheme{Hannover}
%\usetheme{Warsaw}

\title[Ruby]{Ruby Features\\Die spannenden Features von Ruby}
\author{Sven Suska, Thomas Preymesser}
%\institute{Math-linux.com}
\date{2009-Juni-18}
\begin{document}
\lstset{language=Ruby}
\lstset{basicstyle=\small,numbers=left, numberstyle=\tiny, numbersep=5pt}
\begin{frame}
\titlepage
\end{frame}


%\begin{frame}{Einführung}
\frametitle{Was ist Ruby?}

\begin{center}
''Actually, I'm trying to make Ruby natural, not simple.'' 
\end{center}
\begin{flushright}
\emph{Yukihiro ''matz'' Matsumoto}
\end{flushright}



\begin{itemize}
\pause \item Sprache ausdrucksstark\pause, ''natürlich''\pause, flexibel\pause, ''schön''
\pause \item Konsequent objektorientiert (''everthing is an object'')
\pause \item Dynamische Typisierung (''duck typing'')
\pause \item Closures, Metaprogrammierung, offene Klassen
\pause \item Nachteile: (noch) relativ geringe Verbreitung\pause,  langsam\pause, 
              schlechte Thread-Unterstutzung
\end{itemize}
\end{frame}

%\begin{frame}[containsverbatim]
\begin{frame}[fragile]
  \frametitle{Hallo Welt}
  \begin{itemize}[<+->]
  \item nil
  \item 17
  \item i = 123
  \item i + 1000
  \item puts i
  \end{itemize}
  \pause
  Syntaktischer Hauptbestandteil in Ruby sind Ausdrücke ($Expression$s).\\
  Jeder Ausdruck hat beim Ausführen einen Wert.\\
  \pause
  %\item[none]<2->
  \begin{lstlisting}
  text = "welt"
  puts text
  \end{lstlisting}
  \pause
  \lstinline|   puts "hallo " + text + "!"|
  

\begin{comment}
  \pause
  \lstinline|   puts "hallo #{text}!"| \hspace{5em}  (``String-Interpolation'')
  %\end{itemize}
\end{comment}

\end{frame}

\begin{frame}[fragile]
  \frametitle{irb}
  \begin{center}
   {\LARGE irb}\\
   
   \bigskip
   "interactive ruby"
   \bigskip
   \bigskip
   \bigskip
   \bigskip
   
  \end{center}
   shellprompt\$ irb

\end{frame}

\begin{frame}[fragile]
  \frametitle{Ausdrücke und Werte}
  \begin{tabular}[t]{l@{\hspace{5em}$\longrightarrow$}l}
    %{\em Ausdruck}      &   {\em Wert }  \\
    \lstinline|1|       &   \lstinline|1|  \\
    \lstinline|1 + 2|   &   \lstinline|3|  \\
    \lstinline|a = 2|   &   \lstinline|2|  \\
    \lstinline|1+2; 10+20; 17+4|  &   \lstinline|21|  \\
    \lstinline|puts 1+2|   &   \lstinline|nil|  \\
  \end{tabular}
  
  \pause
  Es gibt zwei Arten von Werten.
  
  Unmittelbare Werte (sind immer dasselbe Objekt):
  \begin{lstlisting}
  17
  true
  nil
  :symbol
  \end{lstlisting}
  \pause
  Referenzierte Werte:
  \begin{lstlisting}
  17.0
  "hallo"
  []
  Symbol
  \end{lstlisting}
\end{frame}

\begin{frame}[fragile]
  \frametitle{Methoden}
  Methoden-Definition
  \begin{lstlisting}
  def dreifach(wert)
    wert * 3
  end
  \end{lstlisting}
  \pause
  Methoden-Aufruf
  \begin{lstlisting}
  puts dreifach(2) 
  \end{lstlisting}
  $\longrightarrow$ 6
\end{frame}


\begin{frame}[fragile]
  \frametitle{Methoden-Syntax}
  Klammersetzung bei Methoden-Parametern ist optional:
  \begin{lstlisting}
  def dreifach wert 
    wert * 3
  end
  \end{lstlisting}
  \pause
  \begin{lstlisting}
  puts dreifach(2) 
  puts dreifach 2 
  \end{lstlisting}
  \pause
  \begin{lstlisting}
  puts(dreifach(2)) 
  \end{lstlisting}
  
  \pause
  \begin{lstlisting}
  3 + 4 * 5 
  \end{lstlisting}
  \pause
  \begin{lstlisting}
  3 + (4 * 5)
  \end{lstlisting}
\end{frame}




\begin{frame}
  \frametitle{Threads}
  \lstinputlisting[frame=single,caption={Kuchen backen}]{ch-threads/code/kuchenbacken.rb}
\end{frame}

\begin{frame}
  \frametitle{Threads}
  \lstinputlisting[frame=single,caption={Kuchen backen Ausgabe}]{ch-threads/code/kuchenbacken_ergebnis}
\end{frame}

\begin{frame}
  \frametitle{Threads}
  \begin{itemize}
    \item Jeder Thread l"auft parallel zum Hauptprogramm
    \item Ruby Threads sind Threads innerhalb des Interpreters
    \begin{itemize}
      \item + portabel auf allen Systemen
      \item - keine Mehrprozessor-Systeme unterst"utzt
      \item - System-Calls blockieren andere Threads im Interpreter
    \end{itemize}
  \end{itemize}
\end{frame}

\begin{frame}
  \frametitle{Thread Variablen}
  Variablen in Threads
  \begin{itemize}
    \item externe Variablen
    \item Thread-lokale Variablen
    \item Thread Austausch Variablen \\
          \texttt{Thread.current['zeit'] = 15}
  \end{itemize}
\end{frame}

\begin{frame}
  \frametitle{Exceptions in Threads}
  Exceptions in Threads
  \begin{itemize}
    \item \texttt{Thread.abort\_on\_exception = false} (default)\\
       nicht behandelte Exception beendet diesen Thread
    \item \texttt{Thread.abort\_on\_exception = true}\\
       nicht behandelte Exception beendet alle Threads
  \end{itemize}
\end{frame}


\begin{frame}
  \frametitle{Threads}
  Kontrolle des Thread-Schedulers
  \begin{itemize}
    \item \texttt{Thread.stop} 
    \item \texttt{Thread\#run}
    \item \texttt{Thread.pass}
  \end{itemize}
\end{frame}

\begin{frame}
  \frametitle{Threads}
  \lstinputlisting[frame=single,caption={Kuchen backen}]{ch-threads/code/kuchenbacken2.rb}
\end{frame}

\begin{frame}
  \frametitle{Threads}
  Monitors
  \lstinputlisting[frame=single,caption={counter.rb}]{ch-threads/code/counter.rb}
\end{frame}

\begin{frame}
  \frametitle{Threads}
  Konkurrierende Zugriffe auf Resource
  \begin{itemize}
    \item Problem: \texttt{@counter += 1}\\
    \pause
    \item L"osung: Monitor
  \end{itemize}
\end{frame}

\begin{frame}
  \frametitle{Threads}
  Monitors
  \lstinputlisting[frame=single,caption={counter.rb}]{ch-threads/code/counter2.rb}
\end{frame}

\begin{frame}
  \frametitle{Threads}
  ThreadGroup ???
\end{frame}


\begin{frame}
  \frametitle{Fiber}
  \lstinputlisting[frame=single,caption={aufgaben.rb}]{ch-fiber/code/aufgaben.rb}
\end{frame}



\begin{frame}
  \frametitle{RubyGems}
  Gems sind optional installierbare Bibliotheken f"ur
  Ruby.
  
  Vorbild: CPAN f"ur Perl
 
  Beispiele:
  \begin{itemize}
    \item rmagick f"ur Bildmanipulation
    \item roo zum Lesen von Spreadsheets
    \item rails, Ruby on Rails
    \item ... und mehr als ca. 4000 andere gems
  \end{itemize}
\end{frame}

\begin{frame}
  \frametitle{RubyGems}
  \begin{itemize}
    \item Gems werden standardm"a"sig bei http://www.rubyforge.org gehostet
    \pause
    \item es kann aber auch einfach die <gem>.gem Datei heruntergeladen und anschlie"send installiert werden
    \pause
    \item Gems k"onnen weitere abh"angige Gems automatisch installieren
  \end{itemize} 
\end{frame}

\begin{frame}
  \frametitle{RubyGems}
  Installation:
  \lstinputlisting[language={},frame=single,caption={Gem Installation}]{code/rubygems/installation_result}
\end{frame}

\begin{frame}
  \frametitle{RubyGems}
  Verwendung in Programmen
  \lstinputlisting[frame=single,caption={Anwendung}]{code/rubygems/verwendung.rb}
\end{frame}

\begin{frame}
  \frametitle{RubyGems}
  M"oglichkeiten
  \lstinputlisting[frame=single,caption={Versionen}]{code/rubygems/verwendung2.rb}
\end{frame}

\begin{frame}
  \frametitle{RubyGems}
  \begin{itemize}
  
    \item 'gem query --remote' Liste aller Gems bei Rubyforge
    \item 'gem query --remote --details -n ^roo\$' Listet Details zu Gem 'roo' 
    \item 'gem query --remote --details -n xml' Listet alle Gem Beschreibungen die 'xml' im Namen haben
    \item 'gem install foo'
    \item 'gem install foo --version 1.0.0'
    \item 'gem update'
    \item 'gem uninstall roo'
    \item 'gem cleanup'
    \item 'gem help commands'
    \item 'gem server' (\path{http://localhost:8808/})
  \end{itemize}
\end{frame}

\begin{frame}
  \frametitle{RubyGems}
  Eigene Gems erstellen
  \begin{itemize}
    \item .gemspec Dateien erstellen
    \item Gem 'newgem'
  \end{itemize}
\end{frame}


\begin{frame}
  \frametitle{Automatisierte Tests mit Ruby}
  Automatisierte Tests mit Ruby
  \begin{itemize}
    \item<1-> TestUnit (Standard--Bibliothek in Ruby)
    \item<2-> RSpec (separat zu installieren)
  \end{itemize}
\end{frame}

\begin{frame}
  \frametitle{Automatisierte Tests mit Ruby / TestUnit}
  Automatisierte Tests mit Ruby
  \begin{itemize}
    \item TestUnit (Standard--Bibliothek in Ruby)
    \pause
    \item RSpec 
  \end{itemize}
\end{frame}

\begin{frame}
  \frametitle{Automatisierte Tests mit Ruby / TestUnit}
  Warum automatisierte Tests?
  \begin{itemize}
    \item Leicht auf Knopfdruck aufzurufen
    \item Solide Basis an Tests bei gr"o"seren Anwendungen
    \item Vertrauen bei Modifikation der Anwendung
    \item Test ''dokumentieren'' Verwendung von Code
    \item Test-Driven-Development (TDD)
  \end{itemize}
\end{frame}

\begin{frame}[containsverbatim]
  \frametitle{Automatisierte Tests mit Ruby / TestUnit}
\lstinputlisting[frame=single,caption={konto.rb}]{code/tests/testunit/konto.rb}
\end{frame}

\begin{frame}[containsverbatim]
  \frametitle{Automatisierte Tests mit Ruby / TestUnit}
%\lstinputlisting[frame=single,caption={test$_$konto.rb}]{code/tests/testunit/test_konto.rb}
\end{frame}

\begin{frame}
  \frametitle{Automatisierte Tests mit Ruby / TestUnit}
  \begin{itemize}
    \item Namen der Testmethoden beginnen mit '\texttt{test\_}'
    \item Reihenfolge der Tests ist {\bf nicht} die Reihenfolge im Source
    \item Reihenfolge der Tests ist {\bf nicht} definiert
    \item Aufruf einer einzelnen Testmethode:\\ '\texttt{ruby test\_konto.rb --name test\_anfangssaldo}'
    \item Mehrere Test--Files k"onnen zu Test--Suites zusammengefa"st werden
    \item GUI-Ausgabe statt nur Text m"oglich
    \item Gem ZenTest (autotest)
  \end{itemize}
\end{frame}

\begin{frame}
  \frametitle{Automatisierte Tests mit Ruby / TestUnit}
  Meta-Programming / Reflection
  \begin{itemize}
    \item at\_exit
    \item methods ('test\_')
  \end{itemize}
\end{frame}

\begin{frame}[containsverbatim]
  \frametitle{Automatisierte Tests mit Ruby / TestUnit}
  \lstinputlisting[frame=single,caption={Resultat}]{code/tests/testunit/test_result}
\end{frame}

\begin{frame}
  \frametitle{Automatisierte Tests mit Ruby / TestUnit}
  Test::Unit assertions
  \begin{itemize}
    \item assert
    \item assert\_equal
    \item assert\_not\_equal
    \item assert\_nil
    \item assert\_not\_nil
    \item assert\_kind\_of
    \item assert\_raise(Exception) \{ ... \}
    \item assert\_nothing\_raised(Exception) \{ ... \}
    \item assert\_in\_delta
    \item ...
  \end{itemize}
\end{frame}

\begin{frame}[containsverbatim]
  \frametitle{Automatisierte Tests mit Ruby / TestUnit}
  \lstinputlisting[frame=single,caption={setup / teardown}]{code/tests/testunit/setupteardown.rb}
\end{frame}

\begin{frame}
  \frametitle{Automatisierte Tests mit Ruby / TestUnit}
Optionale Methoden 
\begin{itemize}
  \item setup
  \item teardown
\end{itemize}
werden vor bzw. nach jedem Test aufgerufen
\end{frame}

\begin{frame}
  \frametitle{Logging}
  \begin{itemize}
    \item Klasse Logger in Standard--Library
    \item log4r als Gem
  \end{itemize}
\end{frame}

\begin{frame}
  \frametitle{Logging}
  \lstinputlisting[frame=single,caption={Logging Beispiel}]{code/logging/loggingbeispiel.rb}
\end{frame}

\begin{frame}
  \frametitle{Logging}
  \lstinputlisting[frame=single,caption={Logging Ausgabe}]{code/logging/logging_result}
\end{frame}

\begin{frame}
  \frametitle{Logging}
  Logging-Level:
  \begin{itemize}
    \item FATAL: an unhandleable error that results in a program crash
    \item ERROR: a handleable error condition
    \item WARN:	a warning
    \item INFO:	generic (useful) information about system operation
    \item DEBUG: low-level information for developers 
  \end{itemize}

  M"ogliche Anpassungen:
  \begin{itemize}
    \item \texttt{log.datetime\_format = ''\%H:\%M:\%S''}
    \item \texttt{log = Logger.new(''mein.log'', 5, 10*1024)}
    \item \texttt{logger = Logger.new('foo.log', 'daily')}
    \item \texttt{logger = Logger.new('foo.log', 'weekly')}
    \item \texttt{logger = Logger.new('foo.log', 'monthly')}
 
  \end{itemize}
\end{frame}

\begin{frame}
 \frametitle{Unterschiede 1.8.x / 1.9.x}
 \begin{itemize}
   \item<1->Encoding
   \item<2->String#each \arrow String#each_line
   \item<3->viele Gems funktionieren unter 1.9 noch nicht (www.isitruby19.com)
   \item<4->"ABC"[0] => Ruby 1.8: 65, Ruby 1.9: "A"
   \item<5->Interface f"ur C-Extensions ge"andert
   \item<6->??? Parameter bei Methoden, Bloecken
   \item ... 
 \end{itemize}

\end{frame}

\begin{frame}
  Literatur / Links
  \begin{itemize}
    \item Programming Ruby (Pickaxe Buch) 
    \item Ruby Cookbook
    \item http://www.rubyforen.de
    \item Newsgroup comp.lang.ruby 
    \item Newsgroup de.comp.lang.ruby 
  \end{itemize}
\end{frame}

\end{document}
