 
\documentclass{beamer}

\usepackage[german]{babel}
\usepackage[T1]{fontenc}
\usepackage[latin1]{inputenc} 

\usepackage{listings} 
% \lstset{numbers=left, numberstyle=\tiny, numbersep=5pt} 
 

\usetheme{Warsaw}
\title[Ruby]{Ruby Features\\Die spannenden Features von Ruby}
\author{Sven Suska, Thomas Preymesser}
%\institute{Math-linux.com}
\date{2009-Juni-18}
\begin{document}
\lstset{language=Ruby}
\lstset{numbers=left, numberstyle=\tiny, numbersep=5pt}
\begin{frame}
\titlepage
\end{frame}


\begin{frame}{Einführung}
Was ist Ruby?

''Actually, I'm trying to make Ruby natural, not simple.''
Yukihiro ''matz'' Matsumoto


\begin{itemize}
\pause \item Sprache ausdrucksstark, ''natürlich'', flexibel, ''schon''
\pause \item Konsequent objektorientiert (''everthing is an object'')
\pause \item Dynamische Typisierung (''duck typing'')
\pause \item Closures, Metaprogrammierung, offene Klassen
\pause \item Nachteile: (noch) relativ geringe Verbreitung, langsam, 
\pause \item schlechte Thread-Unterstutzung
\end{itemize}
\end{frame}

\begin{frame}
  \frametitle{Automatisierte Tests mit Ruby}
  Automatisierte Tests mit Ruby
  \begin{itemize}
    \item<1-> TestUnit (Standard--Bibliothek in Ruby)
    \item<2-> RSpec (separat zu installieren)
  \end{itemize}
\end{frame}

\begin{frame}
  \frametitle{Automatisierte Tests mit Ruby}
  Automatisierte Tests mit Ruby
  \begin{itemize}
    \item TestUnit (Standard--Bibliothek in Ruby)
    \pause
    \item RSpec 
  \end{itemize}
\end{frame}

\begin{frame}

 \frametitle{Warum testen?}
  Warum automatisierte Tests?
  \begin{itemize}
    \item Leicht auf Knopfdruck aufzurufen
    \item Solide Basis an Tests bei gr"o"seren Anwendungen
    \item Vertrauen bei Modifikation der Anwendung
    \item Test ''dokumentieren'' Verwendung von Code
    \item Test-Driven-Development (TDD)
  \end{itemize}
\end{frame}

\begin{frame}[containsverbatim]
  \frametitle{Automatisierte Tests mit Ruby}
%  \begin{lstlisting}[fragile,frame=single,caption={konto.rb}]
\lstinputlisting[frame=single,caption={konto.rb}]{code/tests/testunit/konto.rb}
%  \end{lstlisting}
\end{frame}

\begin{frame}[containsverbatim]
  \frametitle{Automatisierte Tests mit Ruby}
  % \begin{lstlisting}[fragile,frame=single,caption={testkonto.rb}]
\lstinputlisting[frame=single,caption={test$_$konto.rb}]{code/tests/testunit/test_konto.rb}
  % \end{lstlisting}
\end{frame}

\begin{frame}
  \frametitle{Automatisierte Tests mit Ruby}
  \begin{itemize}
    \item Namen der Testmethoden beginnen mit '\texttt{test\_}'
    \item Reihenfolge der Tests ist {\bf nicht} die Reihenfolge im Source
    \item Reihenfolge der Tests ist {\bf nicht} definiert
    \item Aufruf einer einzelnen Testmethode:\\ '\texttt{ruby test\_konto.rb --name test\_anfangssaldo}'
    \item Mehrere Test--Files k"onnen zu Test--Suites zusammengefa"st werden
    \item GUI-Ausgabe statt nur Text m"oglich
  \end{itemize}
\end{frame}

\begin{frame}[containsverbatim]
  \frametitle{Automatisierte Tests mit Ruby}
  \lstinputlisting[frame=single,caption={Resultat}]{code/tests/testunit/test_result}
\end{frame}

\begin{frame}
  \frametitle{Automatisierte Tests mit Ruby}
  Test::Unit assertions
  \begin{itemize}
    \item assert
    \item assert\_equal
    \item assert\_not\_equal
    \item assert\_nil
    \item assert\_not\_nil
    \item assert\_kind\_of
    \item assert\_raise(Exception) \{ ... \}
    \item assert\_nothing\_raised(Exception) \{ ... \}
    \item assert\_in\_delta
    \item ...
  \end{itemize}
\end{frame}

\begin{frame}[containsverbatim]
  
  \lstinputlisting[frame=single,caption={setup / teardown}]{code/tests/testunit/setupteardown.rb}
\end{frame}

\begin{frame}
  \frametitle{Automatisierte Tests mit Ruby}
Optionale Methoden 
\begin{itemize}
  \item setup
  \item teardown
\end{itemize}
werden vor bzw. nach jedem Test aufgerufen
\end{frame}

\begin{frame}
  \frametitle{Automatisierte Tests mit Ruby}
 lsjkdflsjfal 
\end{frame}

\begin{frame}
 \frametitle{Unterschiede 1.8.x / 1.9.x}
 \begin{itemize}
   \item<1->Encoding
   \item<2->String#each \arrow String#each_line
   \item<3->viele Gems funktionieren unter 1.9 noch nicht (www.isitruby19.com)
   \item<4->"ABC"[0] => Ruby 1.8: 65, Ruby 1.9: "A"
   \item<5->Interface f"ur C-Extensions ge"andert
   \item ... 
 \end{itemize}

\end{frame}

\end{document}
