 
\documentclass{beamer}

\usepackage[german]{babel}
\usepackage[T1]{fontenc}
\usepackage[latin1]{inputenc} 

\usepackage{listings} 
% \lstset{numbers=left, numberstyle=\tiny, numbersep=5pt} 
 

\usetheme{Warsaw}
\title[Ruby]{Ruby Features\\Die spannenden Features von Ruby}
\author{Sven Suska, Thomas Preymesser}
%\institute{Math-linux.com}
\date{2009-Juni-18}
\begin{document}
\lstset{language=Ruby}
\begin{frame}
\titlepage
\end{frame}


\begin{frame}{Einführung}
Was ist Ruby?

''Actually, I'm trying to make Ruby natural, not simple.''
Yukihiro ''matz'' Matsumoto


\begin{itemize}
\pause \item Sprache ausdrucksstark, ''natürlich'', flexibel, ''schon''
\pause \item Konsequent objektorientiert (''everthing is an object'')
\pause \item Dynamische Typisierung (''duck typing'')
\pause \item Closures, Metaprogrammierung, offene Klassen
\pause \item Nachteile: (noch) relativ geringe Verbreitung, langsam, 
\pause \item schlechte Thread-Unterstutzung
\end{itemize}
\end{frame}

\begin{frame}
  \frametitle{Automatisierte Tests mit Ruby}
  Automatisierte Tests mit Ruby
  \begin{itemize}
    \item<1-> TestUnit (Standard--Bibliothek in Ruby)
    \item<2-> RSpec (separat zu installieren)
  \end{itemize}
\end{frame}

\begin{frame}
  \frametitle{Automatisierte Tests mit Ruby}
  Automatisierte Tests mit Ruby
  \begin{itemize}
    \item TestUnit (Standard--Bibliothek in Ruby)
    \pause
    \item RSpec 
  \end{itemize}
\end{frame}

\begin{frame}

 \frametitle{Warum testen?}
  Warum automatisierte Tests?
  \begin{itemize}
    \item Leicht auf Knopfdruck aufzurufen
    \item Solide Basis an Tests bei gr"o"seren Anwendungen
    \item Vertrauen bei Modifikation der Anwendung
    \item Test ''dokumentieren'' Verwendung von Code
    \item Test-Driven-Development (TDD)
  \end{itemize}
\end{frame}

\begin{frame}[containsverbatim]
  \begin{lstlisting}[fragile,frame=single,caption={konto.rb}]
class Konto
  attr_reader :saldo
  
  def initialize(saldo=0)
    @saldo = saldo
  end
    
  def einzahlen(betrag)
    @saldo = @saldo + betrag
  end
end
  \end{lstlisting}
\end{frame}

\begin{frame}[containsverbatim]
  \begin{lstlisting}[fragile,frame=single,caption={konto.rb}]
class TestKonto < Test::Unit::Testcase
  attr_reader :saldo
  
  def initialize(saldo=0)
    @saldo = saldo
  end
    
  def einzahlen(betrag)
    @saldo = @saldo + betrag
  end
end
  \end{lstlisting}
\end{frame}

\end{document}
