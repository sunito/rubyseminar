\documentclass{beamer}


\usepackage[german]{babel}
\usepackage[T1]{fontenc}
%\usepackage[latin1]{inputenc} 
\usepackage[utf8x]{inputenc}

\usepackage{verbatim} 
\usepackage{listings} 

% \lstset{numbers=left, numberstyle=\tiny, numbersep=5pt} 

%\usetheme{Warsaw}
\usetheme{Hannover}
%\usetheme{PaloAlto}
%\usetheme{JuanLesPins}
%\usetheme{Antibes}
%\usetheme{Shingara}
%\usetheme{Berlin}
%\usetheme{Oxygen}

\usecolortheme{beaver}
% albatross | beaver | beetle |
% crane | default | dolphin |
% dove | fly | lily | orchid |
% rose |seagull | seahorse |
% sidebartab | structure |
% whale | wolverine


\title[Ruby]{Was kann Ruby?}
\author{Sven Suska, Thomas Preymesser}
%\institute{}
\date{2009-Juli-2}




\begin{document}
\lstset{language=Ruby}
\lstset{basicstyle=\small,numbers=none, numberstyle=\tiny, numbersep=5pt}


\begin{frame}[fragile]

  \frametitle{Methoden an Objekten}
  \begin{tikzpicture}[node distance=3cm, auto,>=latex', thick]
    % We need to set at bounding box first. Otherwise the diagram
    % will change position for each frame.
    \path[use as bounding box] (-3,0) rectangle (16,-6);
    
    \path[->]
                  node[klasse] (Person) {Person};
    \path[->]
                  node[klasse, below of=Person, label={[name=lNatPerson]left:}] (NatPerson) {NatPerson}
                  (NatPerson) edge[superc] (Person);
    \path[->]
                  node[objekt, below left of=NatPerson] (heinz) {heinz}
                  (heinz) edge[class] node {class} (NatPerson);
    \pause
    \path[->]
                  node[klasse, right of=heinz] (C-heinz) {Class:heinz}
                  (heinz) edge[eigenc] node {eigenclass} (C-heinz);
    \pause
    \path[->]
                  (heinz.south east) edge[method1r] (C-heinz.west)
                  (C-heinz.west) edge[methods] (NatPerson.west)
                  (NatPerson.west) edge[methods] (Person.west);
\end{tikzpicture}

\begin{comment}
    \pause
    \path[->]
                  (heinz) edge[method1] (NatPerson.west)
                  (NatPerson.west) edge[methods] (Person.west);
    \path[->]
                  node[klasse, right of=NatPerson] (C-NatPerson) {Class:NatPerson}
                  (NatPerson) edge[eigenc] (C-NatPerson);
 
\end{comment}


\end{frame}

\begin{frame}[fragile]

  \frametitle{Methoden an Objekten}
  \begin{tikzpicture}[node distance=3cm, auto,>=latex', thick]
    % We need to set at bounding box first. Otherwise the diagram
    % will change position for each frame.
    \path[use as bounding box] (-3,0) rectangle (10,-2);
    
    \path[->]<1-> node[klasse] (Person) {Person};
    \path[->]<1-> node[klasse, right of=Person] (C-Person) {Class:Person}
                  (Person) edge[eigenc] (C-Person);
    \path[->]<1-> 
                  node[objekt, below left of=Person] (person) {person}
                  node[klasse, right of=person] (C-person) {Class:person}
                  (C-person) edge[superc] (C-Person)
                  (person) edge[class] node {class} (Person)
                            edge[eigenc] node {eigenclass} (C-person);
    \path[->]<1-> node[klasse, below of=Person] (NatPerson) {NatPerson}
                  (NatPerson) edge[superc] (Person);
    \path[->]<1-> node[klasse, right of=NatPerson] (C-NatPerson) {Class:NatPerson}
                  (NatPerson) edge[eigenc] (C-NatPerson);
    \path[->]<1-> 
                  node[objekt, below left of=NatPerson] (heinz) {heinz}
                  node[klasse, right of=heinz] (C-heinz) {Class:heinz}
                  (C-heinz) edge[superc] (C-NatPerson)
                  (heinz) edge[class] node {class} (NatPerson)
                            edge[eigenc] node {eigenclass} (C-heinz);
\end{tikzpicture}
\end{frame}

\begin{frame}[fragile]
  \frametitle{Methoden an Objekten}
  \begin{tikzpicture}[node distance=3cm, auto,>=latex', thick]
    % We need to set at bounding box first. Otherwise the diagram
    % will change position for each frame.
    \path[use as bounding box] (-3,0) rectangle (10,-2);
    
    \path[->]<1-> node[klasse] (Person) {Person};
    \path[->]<1-> node[klasse, right of=Person] (C-Person) {Class:Person}
                  (Person) edge[eigenc] (C-Person);
    \path[->]<1-> 
                  node[objekt, below left of=Person] (person) {person}
                  node[klasse, right of=person] (C-person) {Class:person}
                  (C-person) edge[superc] (C-Person)
                  (person) edge[class] node {class} (Person)
                            edge[eigenc] node {eigenclass} (C-person);
\end{tikzpicture}
\end{frame}

\begin{frame}[fragile]
\begin{comment}
  \begin{displaymath}
  \begin{tikzpicture}[draw,thick, fill=blue!20]
    \xymatrix{
        %\begin{tikzpicture}
        \tikz\node[circle] {Circle};
        %\end{tikzpicture}
        \ar[r]|f \ar[d]|g & B \ar[d]|{g’} \\
        D \ar[r]|{f’}       & C }
  \end{tikzpicture}
  \end{displaymath}
     \path[->]<3-> node[format, right of=dvi] (ps) {.ps file}
                  node[medium, below of=dvi] (screen) {screen}
                  (dvi) edge node {dvips} (ps)
                        edge node[swap] {xdvi} (screen);

    \path[->]<1-> (pdf) edge (screen)
                        edge (print);
    \path[->, draw]<6-> (tex) -- +(0,1) -| node[near start] {pdf\TeX} (pdf);
\end{comment}

  
\end{frame}


\end{document}
