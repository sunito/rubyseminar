\documentclass{beamer}


\usepackage[german]{babel}
\usepackage[T1]{fontenc}
%\usepackage[latin1]{inputenc} 
\usepackage[utf8x]{inputenc}

\usepackage{verbatim} 
\usepackage{listings} 

% \lstset{numbers=left, numberstyle=\tiny, numbersep=5pt} 

%\usetheme{Warsaw}
\usetheme{Hannover}
%\usetheme{PaloAlto}
%\usetheme{JuanLesPins}
%\usetheme{Antibes}
%\usetheme{Shingara}
%\usetheme{Berlin}
%\usetheme{Oxygen}

\usecolortheme{beaver}
% albatross | beaver | beetle |
% crane | default | dolphin |
% dove | fly | lily | orchid |
% rose |seagull | seahorse |
% sidebartab | structure |
% whale | wolverine


\title[Ruby]{Was kann Ruby?}
\author{Sven Suska, Thomas Preymesser}
%\institute{}
\date{2009-Juli-2}


\usepackage{tikz}
\usetikzlibrary{arrows,shapes,decorations}

\usepackage[arrow,matrix,curve]{xy}

\begin{document}
\lstset{language=Ruby}
\lstset{basicstyle=\small,numbers=none, numberstyle=\tiny, numbersep=5pt}

\begin{frame}[fragile]

\tikzstyle{format} = [draw, thin, fill=blue!20]
\tikzstyle{medium} = [ellipse, draw, thin, fill=green!20, minimum height=2.5em]


  \frametitle{Methoden an Objekten}
  \begin{tikzpicture}[node distance=3cm, auto,>=latex', thick]
    % We need to set at bounding box first. Otherwise the diagram
    % will change position for each frame.
    \path[use as bounding box] (-1,0) rectangle (10,-2);
    
    \path[->]<1-> node[format] (Adresse) {Adresse};
    \path[->]<1-> node[format, right of=Adresse] (C-Adresse) {Class:Adresse}
                  (Adresse) edge[bend right] node {eigenclass} (C-Adresse);
    \path[->]<1-> 
                  node[medium, below of=Adresse] (adresse) {adresse}
                  node[medium, below of=C-Adresse] (C-adresse) {Class:adresse}
                  (Adresse) edge node {ps2pdf} (adresse)
                            edge node[swap] {gs} (C-adresse);
\end{tikzpicture}

  \begin{displaymath}
  \begin{tikzpicture}[draw,thick, fill=blue!20]
    \xymatrix{
        %\begin{tikzpicture}
        \tikz\node[circle] {Circle};
        %\end{tikzpicture}
        \ar[r]|f \ar[d]|g & B \ar[d]|{g’} \\
        D \ar[r]|{f’}       & C }
  \end{tikzpicture}
  \end{displaymath}
\begin{comment}
     \path[->]<3-> node[format, right of=dvi] (ps) {.ps file}
                  node[medium, below of=dvi] (screen) {screen}
                  (dvi) edge node {dvips} (ps)
                        edge node[swap] {xdvi} (screen);

    \path[->]<1-> (pdf) edge (screen)
                        edge (print);
    \path[->, draw]<6-> (tex) -- +(0,1) -| node[near start] {pdf\TeX} (pdf);
\end{comment}

  
\end{frame}


\end{document}
