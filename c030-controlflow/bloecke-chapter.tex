\documentclass{beamer}


\usepackage[german]{babel}
\usepackage[T1]{fontenc}
%\usepackage[latin1]{inputenc} 
\usepackage[utf8x]{inputenc}

\usepackage{verbatim} 
\usepackage{listings} 

% \lstset{numbers=left, numberstyle=\tiny, numbersep=5pt} 

%\usetheme{Warsaw}
\usetheme{Hannover}
%\usetheme{PaloAlto}
%\usetheme{JuanLesPins}
%\usetheme{Antibes}
%\usetheme{Shingara}
%\usetheme{Berlin}
%\usetheme{Oxygen}

\usecolortheme{beaver}
% albatross | beaver | beetle |
% crane | default | dolphin |
% dove | fly | lily | orchid |
% rose |seagull | seahorse |
% sidebartab | structure |
% whale | wolverine


\title[Ruby]{Was kann Ruby?}
\author{Sven Suska, Thomas Preymesser}
%\institute{}
\date{2009-Juli-2}




\begin{document}
\lstset{language=Ruby}
\lstset{basicstyle=\small,numbers=none, numberstyle=\tiny, numbersep=5pt, showstringspaces=false}


\begin{frame}[fragile]
  \frametitle{Blöcke sind wie Callbacks}
  Hinter jedem Methodenaufruf kann ein Block stehen.\\
  Das ist ein beliebiges Stück Code ("'Ausdruck"' / $Expression$),
  das entweder in geschweifte Klammern, oder in \path{do ... end}
  eingeschlossen ist.
  
  \pause
  \smallskip
  \begin{lstlisting}
  blocktest do
    puts "mittendrin"
  end
  \end{lstlisting}
  \pause
  
  \begin{lstlisting}
  def blocktest
    puts "vorher"
    yield
    puts "nacher"
  end
  \end{lstlisting}
  
  \begin{tabular}[t]{l@{\hspace{5em}$\longrightarrow$\hspace{5pt}}l}
  \end{tabular}
\end{frame}

\begin{frame}[fragile]
  \frametitle{Blöcke sind vielseitig...}
  Blöcke können Parameter tragen.\\
  
    \pause
  \smallskip
  \begin{lstlisting}
  blocktest do |wert|
    puts "Diesen Wert erhalten: #{wert}"
  end
  \end{lstlisting}
  \pause
  
  \begin{lstlisting}
  def blocktest
    puts "vorher"
    yield "Daten, Daten, Daten"
    puts "nacher"
  end
  \end{lstlisting}
  
  \medskip
  Verwendung für Iteratoren:
  \begin{lstlisting}
  [2, 4, 6].each do |zahl|
    puts zahl * zahl
  end
  \end{lstlisting}
  \pause
  
  \begin{lstlisting}
  \end{lstlisting}
  
  \pause
  \begin{tabular}[t]{l@{\hspace{5em}$\longrightarrow$\hspace{5pt}}l}
  \end{tabular}
\end{frame}

\end{document}
