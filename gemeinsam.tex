\usepackage[german]{babel}
\usepackage[T1]{fontenc}
%\usepackage[latin1]{inputenc} 
\usepackage[utf8x]{inputenc}

\usepackage{verbatim} 
\usepackage{listings} 

% \lstset{numbers=left, numberstyle=\tiny, numbersep=5pt} 
\logo{\includegraphics[width=0.7cm]{../images/ruby.png}}

%\usetheme{Warsaw}
%\usetheme{Hannover}
%\usetheme{PaloAlto}
%\usetheme{JuanLesPins}
%\usetheme{Antibes}
%\usetheme{Shingara}
%\usetheme{Berlin}
%\usetheme{Oxygen}
\usetheme{Singapore}

\usecolortheme{beaver}
% albatross | beaver | beetle |
% crane | default | dolphin |
% dove | fly | lily | orchid |
% rose |seagull | seahorse |
% sidebartab | structure |
% whale | wolverine


\title[Ruby]{Programmiersprache Ruby}
\author{Sven Suska, Thomas Preymesser}
%\institute{}
\date{2009-Juli-2}


\usepackage{tikz}
\usetikzlibrary{arrows,shapes,decorations}


\tikzstyle{klasse} = [rectangle, draw, thin, fill=blue!20 ]
\tikzstyle{objekt} = [ellipse, draw, thin, fill=green!20, minimum height=2.5em]

\tikzset{%
    class/.style={green,>=triangle 45}
    }
\tikzset{%
    eigenc/.style={very thin,green,>=triangle 45}
    }
\tikzset{%
    superc/.style={thick,red,>=open triangle 45}
    }
\tikzset{%
    method1/.style={thin,blue,bend left=45,to reversed->}
    }
\tikzset{%
    method1r/.style={thin,blue,bend right=45,to reversed->}
    }
\tikzset{%
    methods/.style={thin,blue,bend left=45}
    }

\newcommand{\klassebox}[2]{\begin{tabular}{l}{\it #1} \\ \hline \tt{#2}\end{tabular}}

\newcommand{\bls}[1]{{\bf\lstinline{#1}}}
